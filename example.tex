\newpage

\section*{Example Usage of Basic {\LaTeX} Commands}

\subsection*{Where is this Section?}
\paragraph{}Check out \texttt{example.tex} for the contents of this section.

\subsection*{How do I not see this section?}
\paragraph{}This section will be commented out after a while; it will also be hidden if the variable \texttt{showtodo} is set to \texttt{disable} (located in \texttt{./config.sty}). You can also comment out this section by comment out the command in \texttt{./src/\docSrc/index.tex} referencing \texttt{./example.tex}.

\subsection*{References, Hyperlinks, Citations}
\paragraph{}Here is a citation \cite{example2024}. One can add URLs like \url{https://www.youtube.com/watch?v=dQw4w9WgXcQ}, or with different names, such as the following link to an \href{https://www.youtube.com/watch?v=dQw4w9WgXcQ}{informative video on the benefits of perseverance}.

\subsection*{Equations and Formulas}
\paragraph{}An inline equation $x^2+1=0$, or in display form:
\[\int_0^1 (x^2+1)\;\mathrm{d}x\]
see also the \texttt{equation, multline, align} environments from the \texttt{amsmath} package. The site \href{https://detexify.kirelabs.org/classify.html}{Detexify} can be handy.

\subsection*{Include Graphics (Pictures, Charts...)}
\paragraph{}A picture can be included using \texttt{includegraphics} (on Overleaf, one may simply use the snippet for begin figure), and referenced via \texttt{ref}, such as \ref{fig:example}; see the top of this page for the graphics included.
\begin{center}
\begin{figure}
    \centering
    \includegraphics[scale=1]{media/coffee.jpg}
    \caption{a stock photo of coffee making}
    \label{fig:example}
\end{figure}
\end{center}
\subsection*{Ordered or Unordered Lists}
\paragraph{}Ordered lists and unordered list can be given via the \texttt{enumerate, itemize} environments. For example the first few Heilmeier questions:
\begin{enumerate}[label=(\arabic*)]
    \item What are you trying to do? Articulate your objectives using absolutely no jargon.
    \item How is it done today; what are the limits of current practice?
    \item What's new in your approach? Why will it be successful?
    \item Who cares?
    \item[...]
\end{enumerate}
\paragraph{}In short, \texttt{enumerate, itemize, item} are kinda like \texttt{ol, ul, li} in HTML. Appearance of the items can be tuned via packages, such as \texttt{enumitems}.

\subsection*{Todo Notes and Alert Blocks}
\paragraph{}One can also add notes via the \texttt{todonotes} package. For example: 
\todo[
    caption={Example todo},
    color={yellow!40}
]{Finish adding contents to this example.}
\paragraph{}This document has a custom-defined command called \texttt{alertblock} (name taken from bootstrap) using todonotes' functions. The syntax is:
\begin{verbatim}
\alertblock{<alert-type>}{<short-description>}{<long-description>}
\end{verbatim}
\paragraph{}Currently, \texttt{<alert-type>} can take values \texttt{todo, checklist, comment, status}. This command will generate an alert block. For example, the following:
\begin{verbatim}
\alertblock{comment}{Example Comment}%
{An example comment demonstrating what an example comment is.}
\end{verbatim}
\paragraph{}will result in:
\alertblock{comment}{Example}{An example comment demonstrating what an example comment is.}
\paragraph{}One can completely turn-off the \texttt{todonotes} functionalities by setting the \texttt{showtodo} variable to \texttt{disable} in \texttt{config.sty}.

\subsection*{Fontawesome Icons}
\paragraph{}Fontawesome icons can be used in \LaTeX; it is given by the \texttt{fontawesome5} package. For  example, these are some icons for some popular programming languages: \faPython, \faJava, \faJs, \faRust; variants are also possible (if provided), such as for \texttt{faFile}, may use either \faFile, \faFile*~, \faFile[regular]~, or \faFile*[regular]. For more usages, refer to the CTAN documentations.
