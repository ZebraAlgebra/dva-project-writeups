\newpage
\section*{Example Usage of Basic {\LaTeX} Commands}
\paragraph{}Check out \texttt{example.tex} for the contents of this section. This section will be commented out after a while.
\paragraph{}Here is a reference \cite{example2024}. An inline equation $x^2+1=0$, or in display form:
\[\int_0^1 (x^2+1)\;\mathrm{d}x\]
see also the \texttt{equation, multline, align} environments from the \texttt{amsmath} package.
\paragraph{}A picture can be included using \texttt{includegraphics} (on Overleaf, one may simply use the snippet for begin figure), and referenced via \texttt{ref}, such as \ref{fig:example}; see the top of this page for the graphics included.
\begin{center}
\begin{figure}
    \centering
    \includegraphics[scale=1]{media/coffee.jpg}
    \caption{a stock photo of coffee making}
    \label{fig:example}
\end{figure}
\end{center}
\paragraph{}Ordered lists and unordered list can be given via the \texttt{enumerate, itemize} environments. For example the first few Heilmeier questions:
\begin{enumerate}[label=(\arabic*)]
    \item What are you trying to do? Articulate your objectives using absolutely no jargon.
    \item How is it done today; what are the limits of current practice?
    \item What's new in your approach? Why will it be successful?
    \item Who cares?
    \item[...]
\end{enumerate}
\paragraph{}In short, \texttt{enumerate, itemize, item} are kinda like \texttt{ol, ul, li} in HTML. Appearance of the items can be tuned via packages, such as \texttt{enumitems}.
\paragraph{}One can also add notes via the \texttt{todo} package. For example: 
\todo[
    caption={Example todo}
]{Finish adding contents to this example.}
\paragraph{}One can add URLs like \url{https://www.youtube.com/watch?v=dQw4w9WgXcQ}, or with different names, such the following link to an \href{https://www.youtube.com/watch?v=dQw4w9WgXcQ}{informative video on the benefits of perseverance}.
