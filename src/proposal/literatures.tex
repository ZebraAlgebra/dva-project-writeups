
\section{Literature Survey}

% \alertblock{status}{Literature Survey}{This sect}
\alertblock{credit}{Literature Survey Survey}{Names in ascending order.
\begin{center}
    \begin{tabular}{|r|l|l|l|l|l|l|}
         \hline
         &Ryan&Sam&Swathi&Terry&Tim&Tyler\\\hline
         Paper Suggestions&\#1-9&none&\#20-22&\#10-14&\#17-19&\#15-16\\\hline
         Paper Reviews&\#1,3,5,9&\#6,(7,8)&\#20-22&\#10-13&\#17-19&\#(17-19)\\\hline
    \end{tabular}
\end{center}

\begin{itemize}
    \item[\faWrench]Entries in progress: marked with $()$.
    \item[\faSadCry]Orphaned entries: $\#2,4$.
\end{itemize}
}

\alertblock{comment}{GTLib Resource}{We can search for literature using GTech Library \href{https://galileo-gatech.primo.exlibrisgroup.com/discovery/search?query=any,contains,book\%20recommendation\%20clustering&tab=CentralIndex&search_scope=CentralIndex&vid=01GALI_GIT:GT&mfacet=rtype,include,articles,1&mfacet=rtype,include,newspaper_articles,1&lang=en&offset=0}{here}. (Ryan)}

\alertblock{comment}{Citations and bib entries}{Bib entries are added in \texttt{./ref/proposal.bib} (feel free to edit it if you spot errors, or want to add entries, fields); some additional fields including doi/isbn/issn/url infos are added if available. Sample citations for paper \#1 - \#9 are in the list below. The citation keys (the thingy to put inside the \texttt{cite} command) are often of the form (similar to the ones generated by google scholar) "first author's lastname" + "year" + "first word in title". (Sam)}
\alertblock{comment}{Download Papers}{Literatures \#1 - \#12 have been downloaded (legally) and uploaded to the \href{https://gtvault-my.sharepoint.com/:f:/g/personal/rtsedevsuren3_gatech_edu/Eqjd_aliKWNCj8Xpfck44A0B2-t2N89EHWVsAZ4p-Nisqg?e=7NmBTv}{shared folder}.}
\alertblock{comment}{Literature survey expectations}{Taken from \href{https://docs.google.com/document/u/0/d/e/2PACX-1vSlYrMw402tL3F95ay-AaptTdF80UOER-gne_O0kqbuuk6WXrlsjwaYjjS0Jyl95dXYyDLjh9DR1mln/pub?pli=1\#h.z11rqsgxo2dh}{course site}: For each paper, describe  (a) the main idea  (b) why (or why not) it will be useful for your project  (c) its potential shortcomings, that you will try to improve upon.  (Terry)}
\alertblock{todo}{Add literatures and bib entries}{Add paper options for entry \#17 - \#22. For \#6, the bib entry depends on whether we want to cite the whole book or just particular chapters in it.}

\begin{enumerate}[label=(\arabic*)]

    \item \cite{wan2018item} \textbf{Item Recommendation on Monotonic Behavior Chains}
    \begin{itemize}
        \item Main idea: Authors introduce the idea that there are 'monotonic behavior chains' that can be analyzed such as a review weakly signalling that an implied purchase occurred.
        \item Why (or why not) it will be useful: The paper introduces a new algorithm called ChainRec based on this monotic behavior assumption where we know that if there is a review then there was a purchase to recommend similar books. This approach of bringing in the user's behavior might introduce too much complexity on the data science front for our project increasing the scope by too much.
        \item Potential shortcoming we will try to improve on: I don't think there are any shortcommings to improve on for this paper. They are the original collectors of the Goodreads dataset, and their work is thoroughly done. However, they approached it with a complex recommendation algorithm, but I think we can have a simpler recommendation model but a more user friendly interface
    \end{itemize}
    \item \cite{wan2019fine} \textbf{Fine-Grained Spoiler Detection from Large-Scale Review Corpora}
    \item \cite{valdiviezo2022enhanced} \textbf{Enhanced Books Recommendation Using Clustering Techniques and Knowledge Graphs}.
    \begin{itemize}
        \item Main idea: The authors cluster books by using k-means, predict what a user might rate the book, and use the cluster and the predicted book rating to generate a list of recommendations.   Collaborative Filtering is the main approach that is covered in this paper.
        \item Why (or why not) it will be useful: The k-means clustering for recommendations is definitely a simple and reliable place to start for our recommendation system.  The aspect of rating prediction could be incorporated into our project too where the user inputs a rating for a certain book, we check to see how other users have rated that book, see if there is a consensus of highly rated books among those books rated peers, and offer those books to the original user as the recommendation list.
        \item Potential shortcoming we will try to improve on: The authors have not explored the results visualization in detail, so we can improve there.
    \end{itemize}
    \item \cite{lin2023application} \textbf{Application of Data Mining Technology with Improved Clustering Algorithm in Library Personalized Book Recommendation System}
    \item \cite{wang2021exploring} \textbf{Exploring clustering-based reinforcement learning for personalized book recommendation in digital library}
    \begin{itemize}
        \item Main idea of the paper: In the context of library book recommendation systems, the authors wanted to develop a system with the following two issues in their mind -
        \begin{itemize}
            \item Sparsity of user interactions.
            \item Noise in user records.
        \end{itemize}
        RL and DL techniques could help. The HRL (Hiearchical Reinforcement Learning) introduced previously addresses the noise issue, and was used in MOOC recommendations. The authors applied this technique to the book recommendation task, and tweaked it by introducing cluster techniques into it to (CHRL) address the more severe sparseness issue. They also evaluated on two real world datasets and assessed the performance.
        \item Relevance to our project: Help us be aware of the sparsity and noise issues that might also be present in our tasks. Has clear formulas, hyperparameters, and algorithms for CHRL that we may be able to implement it and deploy it if we want to. We can probably also take a look at HRL.
        \item Things we can improve on: Deploy it as a webapp if we want to. Test on bigger dataset.
    \end{itemize}
    \alertblock{comment}{Related Works for \#5 (Ryan)}{Check out the Related Works section of this article for a lot more paper ideas}
    \item \cite{nasraoui2019clustering} \textbf{Clustering Methods for Big Data Analytics: Techniques, Toolboxes and Applications}
    \alertblock{comment}{Related Chapters for \#6 (Ryan)}{There are a couple of potentially relevant chapters in this book}
    \item \cite{jagadev2018collaborative} \textbf{A Collaborative Filtering Approach for Movies Recommendation Based on User Clustering and Item Clustering}
    % \begin{enumerate}
    %     \item Opportunity to skim relevant chapters
    % \end{enumerate}
    \alertblock{comment}{Relevant Chapters for \#7 (Ryan)}{Opportunity to skim relevant chapters}
    \alertblock{todo}{Add Contents (Sam (2/3))}{Literature Review.}
    \begin{itemize}
        \item Main Idea: (text)
        \item Why (or Why Not) it Will be Useful:
        \item Potential Shortcomings We Will Try to Improve On: (text)
    \end{itemize}
    \item \cite{zahra2015novel} \textbf{Novel centroid selection approaches for KMeans-clustering based recommender systems}
    \alertblock{todo}{Add Contents (Sam (3/3))}{Literature Review.}
    \begin{itemize}
        \item Main Idea: (text)
        \item Why (or Why Not) it Will be Useful:
        \item Potential Shortcomings We Will Try to Improve On: (text)
    \end{itemize}
    \item \cite{wang2023comparison} \textbf{Comparison of Multiple Book Recommendation Algorithms After Analysis of User Characteristics Using Big Data}
    \begin{itemize}
        \item Main idea: Hybrid model was the best in terms of classifying users and making recommendations.
        \item Why (or why not) it will be useful: The authors explain they researched Collaborative filtering-based book recommendation algorithm,  Content-based book recommendation algorithm, and Hybrid book recommendation algorithm. Among literature, they claim that collaborative filtering is the most popular, but not necessarily the best. Hybrid is the combo of collaborative filtering and content based recommendations.
        \item Potential shortcoming we will try to improve on: The Goodreads data is much larger than the data set that was used by the authors.
    \end{itemize}
    \item \cite{mupaikwa2025application} \textbf{The Application of Artificial Intelligence and Machine Learning in Academic Libraries}
    \begin{itemize}
        \item Main idea: Libraries are adopting ML and AI techniques as they modernize, but there are many challenges slowing the speed of adoption even though there are opportunities for improvements in operations, efficiency and satisfaction of patrons.
        \item Why (or why not) it will be useful: they explore many uses of data in libraries, including recommender systems that provide filtering, rating and alert options to users based on modeling past reading patterns - support vector machines (SVM) and association rules have been used to recommend books to readers based on loan records and bibliographic information.
        \item Potential shortcoming we will try to improve on: challenges indicated include the low rate of adoption due to concerns about privacy and we can improve on this by creating a recommender without including user specific and identifiable information.
    \end{itemize}
    \alertblock{comment}{Related Works for \#10 (Terry)}{There are a few cited papers that might be useful for exploring the algorithms discussed on a more technical level. Specifically we may want to review Xiao \& Gao (2020), Tsuji et al. (2014), Das \& Islam (2021).}
    \item \cite{monsalve2020autonomous} \textbf{Item Autonomous Recommender System Architecture for Virtual Learning Environments}
    \begin{itemize}
        \item Main idea: This paper explores the recommendation of learning materials specific to the needs of particular students using a hybrid system.
        \item Why (or why not) it will be useful: This paper is mainly a discussion of possible techniques, and they did not develop or test any prototypes. The paper also proposes a deep use of user information including location information, learning styles, and other attributes which would not address privacy concerns around academic and library adoption of AI and ML recommendation systems.
        \item Potential shortcoming we will try to improve on: The idea of a hybrid model is strong, and developing a working prototype is a clear way to improve on the progress of this paper. A model that uses classification to group books based on topics or styles and combines it with the results of a model of prior use (books read or checked out) that does not include user specific data is likely to be viable.
    \end{itemize}
    \item \cite{barsha2023implementing} \textbf{Item Implementing Artificial Intelligence in library services: A review of current prospects and challenges of developing countries}
    \begin{itemize}
        \item Main idea: There are many prospects and challenges to implementing AI and ML solutions in libraries of developing nations.
        \item Why (or why not) it will be useful: The challenges may not be directly applicable to our project, as they cite concerns related to funding, infrastructure and partnerships, which are outside the scope here.
        \item Potential shortcoming we will try to improve on: We are going to be operating with no budget, which could be proof of concept for a very low cost solution to the problems in this paper.
    \end{itemize}
    \item \cite{kumar2021application}
    \textbf{Application of Artificial Intelligence and Machine Learning in Libraries: A Systematic Review}
    \alertblock{comment}{Duplicate Reviews}{Tyler, Terry both both signed up for this entry. The following review is given by Terry. (Sam)}
    \begin{itemize}
        \item Main idea: Use of AI and ML in library information science field has been largely theoretical, and there are clear opportunities for more work on implementation.
        \item Why (or why not) it will be useful: This article confirms that SVM and association rules are commonly considered for book recommendation models.
        \item Potential shortcoming we will try to improve on: We can see clear space to introduce new approaches to book recommendation models, as the space is not focused on implementation.
    \end{itemize}
    \item \cite{qassimi2021folksonomy}
    \textbf{Towards a Folksonomy Graph-Based Context-Aware Recommender System of Annotated Books}
    \alertblock{todo}{Add Contents (Tyler (1/3))}{Literature Review}
    \begin{itemize}
        \item Main Idea: (text)
        \item Why (or Why Not) it Will be Useful:
        \item Potential Shortcomings We Will Try to Improve On: (text)
    \end{itemize}
    \item \cite{kefalas2016taxonomy}
    \textbf{A Graph-Based Taxonomy of Recommendation Algorithms and Systems in LBSNs}
    \alertblock{todo}{Add Contents (Tyler (2/3))}{Literature Review}
    \begin{itemize}
        \item Main Idea: (text)
        \item Why (or Why Not) it Will be Useful: (text)
        \item Potential Shortcomings We Will Try to Improve On: (text)
    \end{itemize}
    \item \cite{sarma2021machinelearning}
    \textbf{Personalized Book Recommendation System Using Machine Learning Algorithm}
    \alertblock{todo}{Add Contents (Tyler (3/3))}{Literature Review}
    \begin{itemize}
        \item Main Idea: (text)
        \item Why (or Why Not) it Will be Useful: (text)
        \item Potential Shortcomings We Will Try to Improve On: (text)
    \end{itemize}
    \item \cite{karpukhin2020dense}
    \textbf{Dense Passage Retrieval for Open-Domain Question Answering}
    \begin{itemize}
        \item Main Idea: The DPR paper introduces a novel deep neural architecture for open domain question answering. The main idea is to compare the embedded BERT features for a query against BERT features for content. A maximum inner product search is performed to return the most relevant texts for a query.
        \item Why (or Why Not) it Will be Useful: We can implement a similar idea to return the most relevant books given a query vector. The query vector can be defined using both textual and relational features.
        \item Potential Shortcomings We Will Try to Improve On: The approach works for open domain QA but it remains to see if it is the correct modeling method for book recommendations and clustering.
    \end{itemize}
    \item
    \cite{maaten2008tsne}
    \textbf{Visualizing Data using t-SNE}
    \begin{itemize}
        \item Main Idea: This paper describes the t-SNE method for visualizing high dimensional data. This method projects the data into lower dimensions (e.g. 2d) and allows for clustering based on this new metric space.
        \item Why (or Why Not) it Will be Useful: We need a way to visualize the books in 2D/3D.
        \item  Potential Shortcomings We Will Try to Improve On: The basic t-SNE method may provide clusters which are misleading. We will determine a tuning strategy to select the best representation in 2d.
    \end{itemize}
    \item
    \cite{wu2018starspace}
    \textbf{StarSpace:
Embed All The Things!}
    \begin{itemize}
        \item Main Idea: This paper introduces a method for embedding arbitrary entities into semantic vectors. It also describes ways of comparing embeddings across types, and combining them into machine learning systems.
        \item Why (or Why Not) it Will be Useful: We may want to embed books into multiple dimensions. For example, we may do a simple text embedding of the contents, and a relational embedding of the graph.
        \item Potential Shortcomings We Will Try to Improve On: We may not need to explicitly model multiple relationships as embeddings. It may be helpful enough to simply consider the idea of multiple embedding types.
    \end{itemize}
    \item \cite{cho2016book} \textbf{Book Recommender System}
    \begin{itemize}
        \item Main Idea: In this paper, the authors have adopted three major approaches for recommendation systems a)content-based b) collaborative c)hybrid
        \item Why (or why not) it will be useful: We think this paper will help us understand the three above approaches and the algorithms that are being used for collaborative filtering approach.
        \item Potential shortcoming we will try to improve: Try to  include a dataset which includes ratings on books by readers.
    \end{itemize}
    \item \cite{li2021personalized} \textbf{Personalized Recommendation Algorithm for books and its implementation}
    \begin{itemize}
        \item Main Idea: n this paper , the authors have chosen the item-based collaborative filtering algorithm to realize personalized recommendation.
        \item Why (or why not) it will be useful: The authors claim that that the algorithm adopted in this paper has better recommendation effect.So we can attempt to try this approach on our datasets and analyze the results.
        \item Potential shortcoming we will try to improve: We don't see any shortcomings for now but may come across them as we work on it.
    \end{itemize}
    \item \cite{zamzami2021analysis} \textbf{Analysis of Library Book Borrower Patterns Using Apriori Association Data Mining Techniques}
    \begin{itemize}
        \item Main Idea: This paper concentrates on using Data Mining techniques mainly apriori algorithm that helps generate association rules for book recommendations and book placement recommendations.
        \item Why (or why not) it will be useful: We think this paper will help us understand a different approach towards building association rules.\\
        Since  placement of books in library is not in scope of our project, that part may not be helpful for now.
        \item Potential shortcoming: we will try to improve: There is very limited literature provided in this article . We will have to do a detailed study on this algorithm if we plan to adopt it.
    \end{itemize}

\end{enumerate}