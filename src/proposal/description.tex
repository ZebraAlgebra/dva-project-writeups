\section{Project Description}
\alertblock{comment}{Project Description Source}{The following is taken from Teams.}
\alertblock{todo}{Project Title}{We might have to think of a project title. This will be included in the document header; we can also include it in this section's header.}

\paragraph{Project Abstract.}Visualize author or book recommendations for user given a book title or author.
 
\paragraph{How is it done today; what are the limits of current practice?}Today, a user most likely Googles something along the lines of “what are books similar to [book title]” or “authors who are similar to [author name].” In looking for other approaches, I found \href{https://www.publishersmarketplace.com/bookscan/about.cgi}{Publisher’s Marketplace}, which shows sales information. There is \href{https://www.publishersmarketplace.com/bookscan/about.cgi}{literature-map} which shows author recommendations. \href{https://www.publishersmarketplace.com/bookscan/about.cgi}{Shepherd} is more analog in its approach of recommendations as they ask authors directly for recommendations. There is also \href{https://www.publishersmarketplace.com/bookscan/about.cgi}{Nathan Rooy’s Visual Book Recommender project} which clusters books on similarity.

The current approaches are good for the respective tasks they try to achieve; however, for a project like Nathan Rooy’s Visual Book Recommender, I believe we can improve the clustering algorithm of books, make the user interface more digestible, and improve the user’s search interaction.
 
\paragraph{What's new in your approach? Why will it be successful?} Our new approach is to make the visualization more sleek in terms of how much information is presented to the user. Maybe we can incorporate new datasets or different data sources to have a better recommendation algorithm.
 
\paragraph{Who cares?}Books are still one of the most popular ways to consume ideas and stories. This visualization approach can also be incorporated into apps like Kindle Books or Libby to help readers determine what book they might check out next. This will be a good complementary tool for any reader.
 
\paragraph{If you're successful, what difference and impact will it make, and how do you measure them (e.g., via user studies, experiments, ground truth data, etc.)?}If successful, the impact will be that more people will read for longer as hopefully they are able to find books they are interested in. If this is done in an app like Libby, then user engagement (read time) can be used as a measure of success.
 
\paragraph{What are the risks and payoffs?}The risks are there are a lot of books being published, especially in the age of self-publication, so we need to determine what is appropriate in terms of scope. The payoff is that this can be a very helpful interface that can be potentially scaled to all different product types like music, movies, plays, etc.

\paragraph{How much will it cost?}It should be low costs, but in a business setting we can consider hosting costs, potential licensing fees, and if there are relevant API fees.
 
\paragraph{How long will it take?}Two months.
 
\paragraph{What are the midterm and final "exams" to check for success?}Midterm would be that the data is all there and the recommendation algorithm seems reasonable. Final would be getting the visualization and interactivity portion finished.
 
\paragraph{How will progress be measured?}Progress will be measured in Trello.





/* Swathi added the below part which could be added to Ryan's part*/

\textbf{What are you trying to do? Articulate your objectives using absolutely no jargon.}
 
    Building and Visualizing  book recommender systems for any user using content-based(summary of a given book)/meta-data 
    based(author/fiction-non-fiction)/collaborative filtering(user ratings etc).
 
\textbf{How is it done today; what are the limits of current practice?}
 
    In current practice, adults and small kids who are not tech-savvy go to the libraries, check the shelf's and then pick books that interests them.
    This way they may miss good books. They also spend long hours in the library going through the aisles.
 
\textbf{What's new in your approach? Why will it be successful?
 }
     With our new approach- we can make personalized user journeys based on age/gender/preferences/reading habits .
     We think it will be very successful amongst senior living and children who can go to the library give their information and the librarian can pull up their personalized recommendation journey thus reducing the time they spend going round the aisles or libraries can mail a copy of their recommendation journey every month thus helping our non tech-savvy senior living.

\textbf{Who cares?}
 
    Books and libraries are still a man's  best friend.  So it will matter to every other person who loves to read  or would love to start reading and are bored to walk through the aisles in the library.
    

\textbf{ If you're successful, what difference and impact will it make, and how do you measure them (e.g., via user studies, experiments, ground truth data, etc.)?}
 
    Technological advancements have placed its footprints in every other industry these days be it healthcare, retail, banking .We feel such personalized journeys and recommendation engines will definitely help the libraries and everyone in general.

\textbf{ What are the risks and payoffs?}
 
    The risk is that the recommendation engine should recommend books based on age especially to children .
     
\textbf{How much will it cost?}
 
    It should be low costs since if this is successful it can be built for each library in every state.
 
\textbf{How long will it take?}
 
    Two months
     
\textbf{What are the midterm and final "exams" to check for success?}
 
    Recommendation engine is reasonable with minor outliers and we are able to provide some visualization.
 
\textbf{How will progress be measured?}
 
    Progress will be measured in Trello.
 